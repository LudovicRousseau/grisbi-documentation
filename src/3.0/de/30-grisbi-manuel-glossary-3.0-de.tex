%%%%%%%%%%%%%%%%%%%%%%%%%%%%%%%%%%%%%%%%%%%%%%%%%%%%%%%%%%%%%%%
% Contents: The glossary entries
% $Id: grisbi-manuel-glossary.tex, v 1.0 2014/02/12 Jean-Luc Duflot
% $Id: grisbi-manuel-glossary.tex, v 3.0 2024/04 Dominique Brochard : update
% $Id: grisbi-manuel-glossary.tex, v 3.0 2024/11 Dominique Brochard :
% - rename file to 30-xxx
% - add Windows and MacOS X entries
%%%%%%%%%%%%%%%%%%%%%%%%%%%%%%%%%%%%%%%%%%%%%%%%%%%%%%%%%%%%%%%% glossary entries file

%example
%%\newglossaryentry{exemple}{name=exemple, description={essai d'une entrée de glossaire}}

\newglossaryentry{Unicode control characters}{name=Unicode control characters, description={Viele Unicode-Zeichen werden verwendet, um die Interpretation oder Anzeige von Text zu steuern, aber diese Zeichen selbst haben keine visuelle oder räumliche Darstellung.  So wird beispielsweise das Null-Zeichen (U+0000 NULL) in C-Programmierumgebungen verwendet, um das Ende einer Zeichenkette anzuzeigen}}
\newglossaryentry{Verschlüsselung}{name=Verschlüsselung, description={see "zu verschlüsseln"}}
\newglossaryentry{zu verschlüsseln}{name=zu verschlüsseln, description={In der Kryptografie bezeichnet die Verschlüsselung eines Dokuments den Prozess der Verschlüsselung von Informationen. Dabei wird die ursprüngliche Darstellung der Informationen, der so genannte Klartext, in eine andere Form, den so genannten Chiffretext, umgewandelt. Im Idealfall können nur befugte Personen einen Chiffretext in einen Klartext zurückverwandeln und auf die ursprünglichen Informationen zugreifen}}
\newglossaryentry{Komprimierung}{name=Komprimierung, description={Unter Datenkompression versteht man die Kodierung von Informationen mit weniger Bits als die ursprüngliche Darstellung. Ein Gerät, das die Datenkompression durchführt, wird üblicherweise als Encoder bezeichnet, ein Gerät, das die Umkehrung des Prozesses (Dekompression) durchführt, als Decoder}}
\newglossaryentry{CSV}{name=CSV, description={Das Dateiformat CSV steht für englisch \lang{Comma-Separated Values} die Werte durch Kommas trennt}}
\newglossaryentry{CVS}{name=CVS, description={\lang{Concurrent Versions System} (CVS) ist ein Software-System zur Versionsverwaltung von Dateien, das hauptsächlich im Zusammenhang mit Software-Quelltext verwendet wird}}
\newglossaryentry{Debian}{name=Debian, description={Debian ist ein gemeinschaftlich entwickeltes freies Betriebssystem. Debian GNU/Linux basiert auf den grundlegenden Systemwerkzeugen des GNU-Projektes sowie dem Linux-Kernel. Debian wurde im August 1993 von Ian \familyname{Murdock} ins Leben gerufen und wird seitdem aktiv weiterentwickelt. Auf Debian basieren viele weitere Linux-Distributionen, von denen Ubuntu die bekannteste istest. Der Name des Betriebssystems leitet sich von den Vornamen des Debian-Gründers Ian \familyname{Murdock} und seiner damaligen Freundin und späteren Ehefrau Debra \familyname{Lynn} ab}}
\newglossaryentry{Linux-Distribution}{name=Linux-Distribution, description={Eine Linux-Distribution ist eine Auswahl aufeinander abgestimmter Software um den Linux-Kernel, bei dem es sich dabei in einigen Fällen auch um einen mehr oder minder angepassten und meist in enger Abstimmung mit Upstream selbst gepflegten Distributionskernel handelt. Distributionen, in denen \lang{GNU}-Programme eine essenzielle Rolle spielen, werden auch als \lang{GNU/Linux-Distributionen} bezeichnet}}
\newglossaryentry{Texteditor}{name=Texteditor, description={Ein Texteditor (von lateinisch textus ‚Inhalt‘ und editor für ‚Herausgeber‘ oder ‚Erzeuger‘) ist ein Computerprogramm zum Bearbeiten von Texten. Der Editor lädt die zu bearbeitende Textdatei und zeigt ihren Inhalt auf dem Bildschirm an, mit der Möglichkeit, durch positionieren eines Cursors einzelne Textzeichen hinzuzufügen oder zu löschen.Ein Texteditor (von lateinisch textus ‚Inhalt‘ und editor für ‚Herausgeber‘ oder ‚Erzeuger‘) ist ein Computerprogramm zum Bearbeiten von Texten. Der Editor lädt die zu bearbeitende Textdatei und zeigt ihren Inhalt auf dem Bildschirm an, mit der Möglichkeit, durch positionieren eines Cursors einzelne Textzeichen hinzuzufügen oder zu löschen. Im Gegensatz zu einem Textverarbeitungssystem und zu Desktop-Publishing-Software (DTP) bietet ein Texteditor in der Regel nur sehr eingeschränkte Layout- und Formatierungsfunktionen an und speichert den Text als reine Textdatei ohne Formatierungen}}
\newglossaryentry{Zeichenkodierung}{name=Zeichenkodierung, description={Eine Zeichenkodierung (englisch Character encoding, kurz Encoding) erlaubt die eindeutige Zuordnung von Schriftzeichen (i. A. Buchstaben oder Ziffern) und Symbolen innerhalb eines Zeichensatzes. In der elektronischen Datenverarbeitung werden Zeichen über einen Zahlenwert kodiert, um sie zu übertragen oder zu speichern}}
\newglossaryentry{Dateinamenserweiterung}{name=Dateinamenserweiterung, description={Die Dateinamenserweiterung (englisch filename extension), auch als Dateinamenerweiterung, Dateierweiterung, Dateiendung oder Dateisuffix bezeichnet, ist der letzte Teil eines Dateinamens und wird gewöhnlich mit einem Punkt abgetrennt. Die Dateiendung wird oft eingesetzt, um das Format einer Datei erkennbar zu machen. Zum Beispiel identifiziert name.txt eine einfache Textdatei}}
\newglossaryentry{Free Software Foundation}{name=Die \lang{Free Software Foundation} (\lang{FSF}, deutsch Stiftung für freie Software) ist eine nichtstaatliche Stiftung (NGO), die als gemeinnützige Organisation 1985 von Richard \familyname{Stallman} mit dem Zweck gegründet wurde, freie Software zu fördern und für diese Arbeit Kapital zusammenzutragen. Die Hauptaufgabe der FSF ist die finanzielle, personelle, technische und juristische Unterstützung des \lang{GNU}-Projekts}}
\newglossaryentry{Dateiformat}{name=Dateiformat, description={Als Dateiformat wird in der Informatik die vom Inhalt abhängige innere Struktur einer Datei bezeichnet. Der Inhalt einer Datei (Bilder, Filme, Grafiken, Musik, Texte, Videos, Zahlzeichen, Zeichnungen) entscheidet über das zu verwendende Dateiformat. In welchem Dateiformat eine Datei gespeichert wurde, lässt sich teilweise an ihrer Dateinamenserweiterung erkennen}}
\newglossaryentry{formateur de texte}{name=formateur de texte, description={est le nom donné aux logiciels destinés à mettre en page du texte, laissant le rédacteur concentré sur son texte, et sans que l'affichage du résultat ne perturbe son activité créatrice. Le résultat sera obtenu ultérieurement par compilation du document source dans le format désiré, par exemple PDF ou PostScript. \LaTeX\space{} est un exemple de formateur de texte. Il s'agit d'un concept opposé à celui du traitement de texte, où rédaction et mise en forme peuvent être créés au cours du même processus}}
\newglossaryentry{Git}{name=Git, description={est un logiciel de gestion décentralisée de versions logicielles. C'est un logiciel libre créé par Linus Torvalds, le créateur du noyau Linux, et distribué selon les termes de la Licence Publique Générale GNU version 2}}
\newglossaryentry{Gnome}{name=Gnome, description={acronyme de \lang{GNU Network Object Model Environment}, est un environnement de bureau libre et convivial dont l'objectif est de rendre accessible l'utilisation du système d'exploitation \lang{GNU} au plus grand nombre; cette interface graphique est populaire sur les systèmes GNU/Linux et fonctionne également sur la plupart des systèmes de type UNIX}}
\newglossaryentry{Gnucash}{name=Gnucash, description={est un logiciel libre de comptabilité personnelle; il n'utilise pas obligatoirement d'extension pour ses fichiers de données, mais peut utiliser les extensions \file{.xac}, \file{.gnc} et \file{.gnucash}}}
\newglossaryentry{GNU Free Documentation License}{name=\lang{GNU Free Documentation License}, description={est le nom anglais officiel, abrégé en \lang{GFDL}, de la Licence de Documentation Libre GNU. C'est une licence relevant du droit d'auteur, produite par la \lang{Free Software Foundation}. Elle a pour but de protéger la diffusion de contenus libres et peut être utilisée par chacun afin de déterminer le mode de diffusion de son \oe uvre. L'objet de cette licence est de rendre un document, écrit sur tout support (manuel, livre, etc.), \og libre \fg{} au sens de la liberté d'utilisation, à savoir : assurer à chacun la liberté effective de le copier ou de le redistribuer, avec ou sans modifications, commercialement ou non}}
\newglossaryentry{GNU General Public License}{name=\lang{GNU General Public License}, description={est le nom anglais officiel, communément abrégé en \lang{GNU GPL}, voire simplement \lang{GPL}, de la Licence Publique Générale GNU. C'est une licence qui fixe les conditions légales de distribution des logiciels libres du projet \lang{GNU}. Richard Stallman et Eben Moglen, deux des grands acteurs de la \lang{Free Software Foundation}, en furent les premiers rédacteurs. Sa dernière version est la \lang{GNU GPL} version 3 publiée le 29 juin 2007}}
\newglossaryentry{GNU/Linux}{name=GNU/Linux, description={\lang{GNU} est un système d'exploitation libre lancé en 1984 par Richard Stallman et maintenu par le projet \lang{GNU}. Son nom est un acronyme récursif qui signifie en anglais \og \lang{GNU's Not UNIX} \fg{} (littéralement, \og GNU n'est pas UNIX \fg{}). Il reprend les concepts et le fonctionnement d'UNIX. Le système \lang{GNU} permet l'utilisation de tous les logiciels libres, pas seulement ceux réalisés dans le cadre du projet \lang{GNU}. GNU/Linux, ou plus familièrement Linux, est un système d'exploitation libre fonctionnant avec le noyau Linux, qui est une implémentation libre du système UNIX respectant les spécifications POSIX. Son nom vient du prénom de son créateur, Linus Torvalds, avec un petit clin d'\oe il à Unix}}
\newglossaryentry{Grisbi}{name=Grisbi, description={Grisbi est un logiciel libre de comptabilité personnelle. L’origine de son nom se trouve ici}}
\newglossaryentry{GSB}{name=.gsb, description={est l'extension donnée aux fichiers des comptes de Grisbi}}
\newglossaryentry{GTK}{name=GTK, description={(The GIMP Toolkit, anciennement GTK+) est un ensemble de bibliothèques logicielles, c'est-à-dire un ensemble de fonctions informatiques permettant de réaliser des interfaces graphiques}}
\newglossaryentry{GZ}{name=.gz, description={est l'extension usuelle des fichiers compressés par le logiciel libre de compression gzip (acronyme de GNU zip). Les logiciels UNIX sont souvent distribués par des fichiers dont l'extension est .tar.gz ou .tgz, appelés \lang{tarball}. Ce sont des fichiers archivés avec le logiciel tar et compressés ensuite avec gzip. Cependant, depuis la fin des années 1990, de plus en plus de logiciels sont distribués par des fichiers d'extension .tar.bz2, archivés avec le logiciel tar et compressés ensuite avec le logiciel bzip2, parce que bzip2 permet de meilleurs taux de compression que gzip, au prix d'un temps de compression plus long}}
\newglossaryentry{HTML}{name=HTML, description={(\lang{HyperText Markup Language}), soit Langage de Balisage Hypertexte, est le format de données conçu pour représenter les pages web. C’est un langage à balises de formatage qui permet d’écrire des liens hypertexte, de structurer sémantiquement et de mettre en forme le contenu des pages, d’inclure des ressources multimédias (dont des images), des formulaires de saisie et des éléments programmables tels que des applets}}
\newglossaryentry{IBAN}{name=IBAN, description={acronyme anglais pour \lang{International Bank Account Number}. C'est un ensemble de nombres qui identifient précisément et de manière unique un compte bancaire, quel que soit l'endroit où il est tenu dans le monde}}
\newglossaryentry{KDE}{name=KDE, description={(K Desktop Environment = Environnement de bureau K)est un projet de logiciel libre historiquement centré autour d'un environnement de bureau pour systèmes UNIX. Ce projet a évolué et il se compose maintenant d'un ensemble de technologies, parmi lesquelles un environnement de bureau et de nombreuses applications, qui sont utilisés principalement avec les systèmes d'exploitation Linux et BSD. Ce projet est également disponible sous Mac~OS~X, quelques autres UNIX (notamment Solaris) ainsi que Windows. KDE est inclus dans la plupart des distributions GNU/Linux les plus populaires}}
\newglossaryentry{Langage C}{name=Langage C, description={Le C est un langage de programmation impératif, qui a été inventé au début des années 1970 pour écrire le système d'exploitation UNIX. Il a conservé de cela une très grande efficacité pour tout ce qui concerne le développement système. Ainsi le noyau de grands systèmes d'exploitation comme Windows et Linux sont développés en grande partie en C. Il est devenu un des langages les plus utilisés, et de nombreux langages plus modernes comme C++, Java et PHP reprennent des aspects de C}}
\newglossaryentry{LaTeX}{name=\LaTeX, description={est un langage et un système de composition de documents. Il s'agit d'une collection de macro-commandes destinées à faciliter l'utilisation du \og processeur de texte \fg{} TeX. Son nom vient de son auteur Leslie Lamport et est l'abréviation de Lamport TeX. On écrit souvent \LaTeX, son logo, ce logiciel permettant ce type de mise en forme}}
\newglossaryentry{liens hypertexte}{name=liens hypertexte, description={ou hyperliens, ou simplement liens sont de références dans un système hypertexte, permettant de passer automatiquement d'un document consulté à un document lié, ou d’une partie à une autre dans un même document. Les liens hypertexte sont notamment utilisés dans Internet pour permettre le passage d'une page Web à une autre d'un seul clic de souris}}
\newglossaryentry{Licence Publique Generale GNU}{name=Licence Publique Générale GNU, description={est le nom en français de la \lang{GNU General Public License} en anglais, communément abrégé en \lang{GPL}}}
\newglossaryentry{Licence de Documentation Libre GNU}{name=Licence de Documentation Libre GNU, description={est le nom en français de la \lang{GNU Free Documentation License} en anglais, abrégé en \lang{GFDL}}}
\newglossaryentry{locale}{name=locale, description={Les paramètres régionaux, aussi appelés \lang{locales} en anglais, sont un ensemble de définitions de textes et de formats utiles à la régionalisation du logiciel. Ils permettent au logiciel d’afficher les données selon les attentes culturelles et linguistiques propres à la langue et au pays de l’utilisateur: le type de virgule, la représentation des chiffres, le format de la date et de l'heure, les unités monétaires, l'encodage par défaut, l'ordre alphabétique des lettres (qui peut différer selon les régions), etc}}
\newglossaryentry{logiciel libre}{name=logiciel libre, description={Un logiciel libre est un logiciel dont l'utilisation, l'étude, la modification et la duplication en vue de sa diffusion sont possibles techniquement et permises légalement. Ces droits sont le plus souvent définis par une licence. Par exemple, Grisbi est placé sous la licence \lang{GPL} (\lang{General Public License}) ou Licence Publique Générale GNU}}
\newglossaryentry{methodes de saisie}{name=méthodes de saisie, description={Ce sont des manières alternatives de saisir du texte pour obtenir des caractères de différents environnements linguistiques}}
\newglossaryentry{OFX}{name=OFX, description={(\lang{Open Financial Exchange}) est un format ouvert utilisé par des institutions financières et des éditeurs de logiciels; il s'agit d'un formatage XML de données financières}}
\newglossaryentry{partition}{name=partition, description={L'espace de stockage d'un disque dur peut être fractionné en plusieurs partitions. Celles-ci apparaissent au système d'exploitation comme des disques (ou volumes) séparés. Chaque partition possède son propre système de fichiers, qui permet de stocker les fichiers. Les systèmes de fichiers les plus connus sont FAT32, NTFS, ext2, ext3 et ext4}}
\newglossaryentry{PDF}{name=PDF, description={(\lang{Portable Document Format}) est un langage de description de page créé par la société Adobe Systems et dont la spécificité est de préserver la mise en forme d’un document (polices de caractères, images, objets graphiques, etc.) telle qu'elle a été définie par son auteur, et cela quels que soient le logiciel, le système d'exploitation et l'ordinateur utilisés pour l’imprimer ou le visualiser}}
\newglossaryentry{plan comptable}{name=plan comptable, description={C'est l'ensemble des règles d'évaluation et de tenue des comptes qui constitue la norme de la comptabilité française. Le plan de comptes, c'est-à-dire la liste ordonnée des comptes, est un des éléments du plan comptable. C'est à tort que le langage usuel réduit souvent le plan comptable au seul plan de comptes}}
\newglossaryentry{PNG}{name=PNG, description={(\lang{Portable Network Graphics}) est un format ouvert d’images numériques qui a été créé pour remplacer le format GIF, à l’époque propriétaire et dont la compression était soumise à un brevet. Le PNG est un format non destructeur spécialement adapté pour publier des images simples comprenant des aplats de couleurs}}
\newglossaryentry{portage}{name=portage, description={Un portage est l'adaptation d'un programme dans un système d'exploitation autre que celui pour lequel il a été créé à l'origine}}
\newglossaryentry{PostScript}{name=PostScript, description={est un langage informatique spécialisé dans la description de pages, mis au point par la société Adobe Systems. Ce langage interplateformes permet d'obtenir un fichier unique comportant tous les éléments décrivant la page (textes, images, polices, couleurs, etc.). PostScript est devenu pratiquement un standard, la plupart des imprimantes laser haut de gamme peuvent traiter directement le format PostScript}}
\newglossaryentry{QIF}{name=QIF, description={(\lang{Quicken Interchange Format}) est un format d'échange de données financières; c'est le format utilisé par le logiciel Quicken et aussi par plusieurs banques en ligne}}
\newglossaryentry{RedHat}{name=Red Hat, description={est une société multinationale d'origine états-unienne éditant la distribution GNU/Linux éponyme. Elle est l’une des entreprises dédiées aux logiciels Open Source les plus importantes et les plus reconnues. Elle constitue également le premier distributeur du système d’exploitation GNU/Linux}}
\newglossaryentry{Slackware}{name=Slackware, description={est une distribution Linux qui, à la différence d'autres distributions populaires, a longtemps été maintenue par une seule personne. Elle est connue pour suivre au mieux la \og philosophie Unix \fg{} et se veut être une distribution légère, rapide et sans fioritures; elle est fort appréciée sur les serveurs}}
\newglossaryentry{SVG}{name=SVG, description={(\lang{Scalable Vector Graphics}, soit en français \og graphique vectoriel adaptable \fg{}), est un format de données conçu pour décrire des ensembles de graphiques vectoriels, et est basé sur XML}}
\newglossaryentry{tri}{name=tri, description={Faire un tri signifie, en informatique, organiser une collection de données selon un ordre déterminé et suivant un critère défini, appelé clé de tri}}
\newglossaryentry{tri primaire}{name=tri primaire, description={est le terme employé dans Grisbi pour désigner un tri basé sur une clé primaire de tri, c'est-à-dire un tri de premier niveau. S'il n'y a que le tri primaire sans tri secondaire, il s'agit alors d'un simple tri, basé sur un seul critère de tri}}
\newglossaryentry{tri secondaire}{name=tri secondaire, description={est le terme employé dans Grisbi pour désigner un tri basé sur une clé secondaire de tri, c'est-à-dire un tri de deuxième niveau. Par exemple, vous pouvez faire un tri primaire sur la date de valeur, et un tri secondaire sur le tiers}}
\newglossaryentry{URL}{name=URL, description={abréviation de l'anglais \lang{Uniform Resource Locator}, littéralement \og localisateur uniforme de ressource \fg, souvent appelé \og adresse web \fg, désigne une chaîne de caractères utilisée pour adresser les ressources du World Wide Web : document HTML, image, son, forum Usenet, boîte aux lettres électronique, entre autres}}
\newglossaryentry{UTF-8}{name=UTF-8, description={(\lang{UCS Transformation Format 8 bits}) est un codage informatique de caractères, conçu pour coder l'ensemble des caractères internationaux d'Unicode, tout en restant compatible avec la norme ASCII limitée à l'anglais}}
\newglossaryentry{XML}{name=XML, description={(\lang{eXtensible Markup Language}), soit en français \og langage extensible à balises \fg{}, est un langage informatique de balisage générique qui dérive du langage SGML}}
\newglossaryentry{Mac OS X}{name=Mac OS X, description={est un système d’exploitation partiellement propriétaire développé et commercialisé par Apple depuis 1998, dont la version macOS Sequoia (version 15) a été lancée le 16 septembre 2024 pour le grand public}}
\newglossaryentry{Windows}{name=Windows, description={(littéralement « Fenêtres » en anglais) est une gamme de systèmes d’exploitation propriétaires développés par Microsoft. La première version de Windows, en 1985, n'était qu'une interface graphique pour MS-DOS utilisé sur les ordinateurs IBM. Ont suivis les versions 2, 3, 95 (sortie en 1995 et vendue préinstallée sur la quasi-totalité des ordinateurs personnels, du à de très nombreux accords d'exclusivité passé avec les constructeurs d'ordinateurs leur interdisant d'installer un autre système sous peine de sanctions financières), XP, Vista, 7, 8 et 10. La version 11 est la version actuelle en 2024}}

%% end of glossary entries
